% start preamble -------------------------------------------------------------
\documentclass{article}
\usepackage{amsmath, amsthm, amssymb, amsfonts}
\usepackage{thmtools}
\usepackage{graphicx}
\usepackage{setspace}
\usepackage{geometry}
\usepackage{float}
\usepackage{hyperref}
\usepackage[utf8]{inputenc}
\usepackage[english]{babel}
\usepackage{framed}
\usepackage[dvipsnames]{xcolor}
\usepackage{tcolorbox}
\usepackage{ dsfont }
\usepackage[math]{cellspace}
\usepackage{forest}
\usepackage{tikz-qtree}
\usepackage { cmll }
\usepackage{ tipa }

\setlength\cellspacetoplimit{3pt}
\setlength\cellspacebottomlimit{3pt}
\colorlet{LightGray}{White!90!Periwinkle}
\colorlet{LightOrange}{Orange!15}
\colorlet{LightGreen}{Green!15}

\graphicspath{ {./pictures/} }

\newcommand{\HRule}[1]{\rule{\linewidth}{#1}}
\newcommand*\xor{\mathbin{\oplus}}
% end preamble -------------------------------------------------------------------

\begin{document}

% ------------------------------------------------------------------------------
% Cover Page and ToC
% ------------------------------------------------------------------------------

\title{ \normalsize \textsc{}
		\\ [2.0cm]
		\HRule{1.5pt} \\
		\LARGE \textbf{\uppercase{ Programmiertestat 01 }
        \HRule{2.0pt} \\ [0.6cm] \LARGE{ Rechnerorganisation } \vspace*{10\baselineskip}}
		}
\date{Juni 2023}
\author{\textbf{} \\
		Name: Hannes Albert}

\maketitle
\setlength\leftskip{1cm}

Bemerkung: Für das folgende Testat wurden ausschließlich folgende Quellen verwendent: Die Quellen, welche auf 
dem Aufgabenblatt vorgegeben wurden, zusammen mit der im Moodle-Kurs bereitgestellten 
Pdf $Eingabe(scanf)_Ausgabe(printf).pdf$ .

\section{1.1}
Indem wir die Ausdrücke aus dem Quelltext ineinander einsetzen und 
Umklammern erhalten wir folgenden Ausdruck:
\begin{align*}
    (a \with  b) \text{ } \hat{} \text{ } \sim c * d - (e | f) \\ 
    \overset{Klammern}{\rightarrow} ((a \with b) \text{ }  \hat{} \text{ } (\sim c * d)) - (e | f)
\end{align*}

\section{1.2}
\scalebox{3}{
    \begin{tikzpicture}
        \tikzset{edge from parent/.style={draw,edge from parent path={(\tikzparentnode.south)-- +(0,-8pt)-| (\tikzchildnode)}}}
        \Tree [.-
        [.$\hat{}$ 
        [.$\with$ [.a ] [.b ] ]
        [.* [.$\sim$ [.c ] ]
        [.d ] ] ]
        [.$|$ 
        [.e ]
        [.f ] ] ]
\end{tikzpicture}}
\begin{enumerate}
    \item Den Blattknoten a, b, c, d, e, f wird jeweils der Wert 1 zugewiesen.
    \item Da der Knoten $\sim$ nur ein Kind hat, besitzt dieser 
        Knoten ebenfalls den Wert 1.
    \item Da $\with$, * und $|$ jeweils zwei Kinder mit dem Wert 1 haben,
        besitzten alle dieser Knoten den Wert zwei (1 + 1 = 2).
    \item Da $\hat{}$ zwei Kinder hat, welche jeweils den Wert 2 
        besitzen, hat $\hat{}$ den Wert 3 (2 + 1 = 3).
    \item - hat zwei Kinder. Das erste besitzt den Wert 3 und das 
        zweite den Wert 2. Somit gilt: - = max( $\hat{}$, $|$) = 3.
    \item Da - die Root des Baumes ist terminiert der Algorithmus und die 
        Ershov-Zahl für den gegebenen Baum ist 3.
\end{enumerate}
\end{document}

