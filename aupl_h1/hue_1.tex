\documentclass{article}
\usepackage{graphicx} % Required for inserting images
\usepackage{amsmath}
\usepackage{tipa}
\usepackage{dsfont}
\title{Aupl Hausuebung 1}
\author{Hannes Albert}
\date{April 2023}

\begin{document}

\maketitle

\section{H1}
\setlength\leftskip{1cm}
a) \\
\begin{tabular}[h]{c|c|c|c|c}
    p & q & r & ($\neg$p $\wedge$ $\neg$ q) & ( $\neg$ p $\wedge$ $\neg$q) 
    $\rightarrow$ (p $\vee$ ($\neg$ q $\wedge$ r)) \\
    \hline
    0 & 0 & 0 & 1 & 0 \\
    0 & 0 & 1 & 1 & 1 \\
    0 & 1 & 0 & 0 & 1 \\
    0 & 1 & 1 & 0 & 1 \\
    1 & 0 & 0 & 0 & 1 \\
    1 & 0 & 1 & 0 & 1 \\
    1 & 1 & 0 & 0 & 1 \\
    1 & 1 & 1 & 0 & 1 \\
\end{tabular}

\bigskip
Die Formel ist erfüllbar, da sich mindestens eine Eins in der letzten Spalte
befindet, allerdings nicht allgemeingültig, da sich in der ersten Zeile der letzten 
Spalte eine Null befindet. 
\bigskip \\ 
b)  Indem wir aus der Tabelle die DNF ablesen erhalten wir folgende
    Formel $\varphi$ für die Wahrheitstafel:
    \[
        \varphi = (\neg p \wedge \neg q \wedge \neg r) \vee
        (\neg p \wedge q \wedge \neg r) \vee
        (p \wedge \neg q \wedge r) \vee
        (p \wedge q \wedge \neg r) 
    \]  
\bigskip 
c)  Zunächste erstellen wir eine passende Wahrheitstabelle: 
\[
\begin{tabular}[h]{c|c|c|c}
    p & q & r & $\varphi$ \\ \hline
    0 & 0 & 0 & 1 \\ 
    0 & 0 & 1 & 1 \\
    0 & 1 & 0 & 1 \\ 
    0 & 1 & 1 & 0 \\ 
    1 & 0 & 0 & 1 \\ 
    1 & 0 & 1 & 0 \\ 
    1 & 1 & 0 & 0 \\ 
    1 & 1 & 1 & 0 \\
\end{tabular}
\]
Somit lässt sich folgende DNF ablesen:
\[
    \varphi = (\neg p \wedge \neg q \wedge \neg r) \vee
    (\neg p \wedge \neg q \wedge r) \vee
    (\neg p \wedge q \wedge \neg r) \vee
    (p \wedge \neg q \wedge \neg r)
\]
\bigskip
d) Wie in c) erstellen wir eine passende Wahrheitstabelle
\[
    \begin{tabular}[h]{c|c|c|c|c}
        p & q & r & s & $\varphi$ \\ \hline
        0 & 0 & 0 & 0 & 0 \\  
        0 & 0 & 0 & 1 & 1 \\
        0 & 0 & 1 & 0 & 1 \\ 
        0 & 0 & 1 & 1 & 0 \\ 
        0 & 1 & 0 & 0 & 1 \\ 
        0 & 1 & 0 & 1 & 0 \\ 
        0 & 1 & 1 & 0 & 0 \\ 
        0 & 1 & 1 & 1 & 1 \\ 
        1 & 0 & 0 & 0 & 1 \\
        1 & 0 & 0 & 1 & 0 \\
        1 & 0 & 1 & 0 & 0 \\ 
        1 & 0 & 1 & 1 & 1 \\ 
        1 & 1 & 0 & 0 & 0 \\ 
        1 & 1 & 0 & 1 & 1 \\ 
        1 & 1 & 1 & 0 & 1 \\ 
        1 & 1 & 1 & 1 & 0 \\ 
    \end{tabular}
\]
Damit lässt sich folgende DNF ablesen:
\begin{align*}
    \varphi = (\neg p \wedge \neg q \wedge \neg r \wedge s) \\ 
    \vee (\neg p \wedge \neg q \wedge r \wedge \neg s) \\
    \vee (\neg p \wedge q \wedge \neg r \wedge \neg s) \\
    \vee (\neg p \wedge q \wedge r \wedge s) \\
    \vee (p \wedge \neg q \wedge \neg r \wedge \neg s) \\
    \vee (p \wedge \neg q \wedge r \wedge s) \\
    \vee (p \wedge q \wedge \neg r \wedge s) \\
    \vee (p \wedge q \wedge r \wedge \neg s) 
\end{align*}

\section{H1.2}
a) \\ 
Mithilfe von Beispiel 2.4 bekommen wir folgendes: 
\begin{align*}
    p \oplus q = (\neg p \wedge q) \vee (p \wedge \neg q) \\
               = \neg\neg((\neg p \wedge q) \vee (p \wedge \neg q)) \\
               = \neg(\neg((\neg p \wedge q) \vee (p \wedge \neg q))) \\
               = \neg(\neg(\neg p \wedge q) \wedge \neg(p \wedge \neg q)) 
\end{align*}

\[ 
    p \downarrow q = \neg p \wedge \neg q
\]

b) \\ 
Um dies zu zeigen, wir, dass $\downarrow$ ein anderes vollständiges 
Junktorensystem [$\neg$, $\wedge$] darstellen kann: 
\begin{align*}
   \neg p = p \downarrow p \\
   p \wedge q = (p \downarrow p) \downarrow (q \downarrow q)
\end{align*}
c) \\ 
\begin{tabular}[h]{c|c|c|c}
    p & q & p $\oplus$ q & q $\oplus$ p \\ \hline
    0 & 0 & 0 & 0 \\ 
    0 & 1 & 1 & 1 \\ 
    1 & 0 & 1 & 1 \\ 
    1 & 1 & 0 & 0 \\
\end{tabular}
\quad
\begin{tabular}[h]{c|c|c|c|c}
    p & q & r & (p $\oplus$ q) $\oplus$ r & p $\oplus$ (q $\oplus$ r) \\ \hline
    0 & 0 & 0 & 0 & 0 \\
    0 & 0 & 1 & 1 & 1 \\
    0 & 1 & 0 & 1 & 1 \\ 
    0 & 1 & 1 & 0 & 0 \\ 
    1 & 0 & 0 & 1 & 1 \\ 
    1 & 0 & 1 & 0 & 0 \\ 
    1 & 1 & 0 & 0 & 0 \\ 
    1 & 1 & 1 & 1 & 1 \\
\end{tabular}
\\
Wie in den Tabellen zu erkennen ist sind gilt sowohl kommutativität 
als auch Assoziativität 

d) \\
Ein vollständiges Junktorensystem wie z.B. in b) funktioniert mit $\oplus$
nicht, da z.B. $\neg$ sich nicht darstellen lässt.
Dafür müsste nähmlich für ein p gelten:
\[
    \exists n \in \mathds{Z} | \underbrace{p \oplus p ... \oplus p}_{\substack{n - mal}} = \neg p
\]
Da kein n existiert, so dass die obige Gleichung für p = 0 gilt folgt,
dass es sich bei [$\oplus$] um kein vollständiges Junktorensystem handelt.

\section{H1.3}
\end{document}
